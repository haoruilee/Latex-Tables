\documentclass{article}
\usepackage[utf8]{inputenc}
\usepackage[UTF8]{ctex}
\usepackage{booktabs} %三线表需要加载宏包{booktabs}
\usepackage{diagbox}
\usepackage{multirow}
% 斜线表加载宏包{diag}和{multirow}


\title{Latex tables}
\author{li.haor }
\date{April 2020}

\begin{document}

\maketitle

\section{Basic Table}
%%%%%%%%%%%%%%%%%%%%%%%%%%%%%%%%%%%%%%%%%%%%%%%
\begin{tabular}{cc}%一个c表示有一列,格式为居中显示(center)
(1,1)&(1,2)\\%第一行第一列和第二列  中间用&连接
(2,1)&(2,2)\\%第二行第一列和第二列  中间用&连接
\end{tabular}
%%%%%%%%%%%%%%%%%%%%%%%%%%%%%%%%%%%%%%%%%%%%%%%%%%
\begin{tabular}{|c|c|}% 通过添加 | 来表示是否需要绘制竖线
\hline  % 在表格最上方绘制横线
(1,1)&(1,2)\\
\hline  %在第一行和第二行之间绘制横线
(2,1)&(2,2)\\
\hline % 在表格最下方绘制横线
\end{tabular}
%%%%%%%%%%%%%%%%%%%%%%%%%%%%%%%%%%%%%%%%%%%%%%%%
\begin{tabular}{|l|c|r|} %l(left)居左显示 r(right)居右显示 c居中显示
\hline 
Name&Steve&Bill\\
\hline  
Matlab&Mathmatica&Maple\\
\hline 
\end{tabular}
%%%%%%%%%%%%%%%%%%%%%%%%%%%%%%%%%%%%%%%%%%%%%%%%%%
\section{Three line Table}
%%%%%%%%%%%%%%%%%%%%%%%%%%%%%%%%%%%%%%%%%
\begin{tabular}{ccc}
\hline
姓名& 学号& 性别\\
\hline
Steve Jobs& 001& Male\\
Bill Gates& 002& Female\\
\hline
\end{tabular}
%%%%%%%%%%%%%%%%%%%%%%%%%%%%%%%%%%%%%%%%%%%

\begin{tabular}{ccc}
\toprule  %添加表格头部粗线
姓名& 学号& 性别\\
\midrule  %添加表格中横线
Steve Jobs& 001& Male\\
Bill Gates& 002& Female\\
\bottomrule %添加表格底部粗线
\end{tabular}
%%%%%%%%%%%%%%%%%%%%%%%%%%%%%%%%%%%%%%%
\begin{table}[!htbp]
\centering
\caption{这是一张三线表}\label{tab:aStrangeTable}%添加标题 设置标签
\begin{tabular}{ccc}
\toprule
姓名& 学号& 性别\\
\midrule
Steve Jobs& 001& Male\\
Bill Gates& 002& Female\\
\bottomrule
\end{tabular}
%\caption{这是一张三线表}\label{tab:aStrangeTable}  标题放在这里也是可以的
\end{table}
%%%%%%%%%%%%%%%%%%%%%%%%%%%%%%%%%%%%%%%%
\begin{table}[!htbp]
\centering
\begin{tabular}{|c|c|c|}
\hline
\multicolumn{3}{|c|}{学生信息}\\ % 用\multicolumn{3}表示横向合并三列 
                        % |c|表示居中并且单元格两侧添加竖线 最后是文本
\hline
姓名&学号&性别\\
\hline
Jack& 001& Male\\
\hline
Angela& 002& Female\\
\hline
\end{tabular}
\caption{这是一张三线表}
\end{table}
%%%%%%%%%%%%%%%%%%%%%%%%%%%%%%%%%%%%%%%
\begin{table}[!htbp]
\centering
\begin{tabular}{|c|c|c|c|c|c|c|} %表格7列 全部居中显示
\hline
\multicolumn{7}{|c|}{事件}\\  %横向合并7列单元格  两侧添加竖线
\hline
\multirow{4}*{策略}&50&0&100&200&300&300\\  %纵向合并4行单元格 
\cline{2-7}  %为第二列到第七列添加横线
&100&100&0&100&200&200\\
\cline{2-7}
&150&200&100&0&100&200\\
\cline{2-7}
&200&300&200&100&0&300\\
\hline
\end{tabular}
\end{table}
%%%%%%%%%%%%%%%%%%%%%%%%%%%%%%%%%%%%%%%%%%
\begin{table}[!htbp]
\centering
\begin{tabular}{|c|c|c|c|c|c|c|}
\hline

\multicolumn{2}{|c|}{ \multirow{2}*{$S_i$} }& \multicolumn{4}{c|}{事件} &\multirow{2}*{max}\\
\cline{3-6}
\multicolumn{2}{|c|}{}&50&100&150&200&\\
\hline
\multirow{4}*{策略}&50&0&100&200&300&300\\
\cline{2-7}
&100&100&0&100&200&200\\
\cline{2-7}
&150&200&100&0&100&200\\
\cline{2-7}
&200&300&200&100&0&300\\
\hline
\end{tabular}
\end{table}
%%%%%%%%%%%%%%%%%%%%%%%%%%%%%%%%%%%%%%%%%%%%%%%

\section{Diagonal Table}

\begin{table}[!htbp]
\centering
\begin{tabular}{|c|c|c|c|}
\hline
\diagbox{甲}{$\alpha_{i,j}$}{乙}&$\beta_1$&$\beta_2$&$\beta_3$\\ %添加斜线表头
\hline
$\alpha_1$&-4&0&-8\\
\hline
$\alpha_2$&3&2&4\\
\hline
$\alpha_3$&16&1&-9\\
\hline
$\alpha_4$&-1&1&7\\
\hline
\end{tabular}
\end{table}
%%%%%%%%%%%%%%%%%%%%%%%%%%%%%%%%%%%%%%%%%%%%%%
%解决斜线对齐
%参数\diagbox[innerwidth=2cm](参数大小取决于列宽度)
\begin{table}[!htbp]
  \centering
  \begin{tabular}{|c|c|c|c|c|c|c|}
   \hline
   \multicolumn{2}{|c|}{\multirow{2}*{\diagbox[innerwidth=2cm]{$S_i$}{$\lambda_i$}}}& \multicolumn{4}{c|}{事件} &\multirow{2}*{max}\\
   \cline{3-6}
   \multicolumn{2}{|c|}{}&50&100&150&200&\\
   \hline
   \multirow{4}*{策略}&50&0&100&200&300&300\\
   \cline{2-7}
   &100&100&0&100&200&200\\
   \cline{2-7}
   &150&200&100&0&100&200\\
   \cline{2-7}
   &200&300&200&100&0&300\\
   \hline
  \end{tabular}
 \end{table}
\end{document}
%%%%%%%%%%%%%%%%%%%%%%%%%%%%%%%%%%%%%%

\end{document}
